\section{Введение}

    В настоящее время задачи экологии имеют большое значение. Важно научиться применять методы для анализа математических моделей различных экологических систем. 
    
    Одна из основных задач экологии --- изучение структуры системы и то, как она функционирует, поиск закономерностей. В качестве инструмента для анализа систем можно использовать методы из различных разделов математики, в частности нелинейной динамики.
    
    В данной работе предложены некоторые выкладки по анализу дискретной модели Хасселя с эффектом Олли \cite{densityDependenceInSingleSpeciesPopulations}, \cite{inequalityInResourceAllocationAndPopulationDynamicsModels}. Подобные модели широко используется в качестве общих моделей динамики популяции с дискретным временем и наличием конкуренции за ресурсы и внутривидовой конкуренции. Так же их используют для исследования различных явлений в динамике популяций.