\section{Введение}
    
    В настоящее время задачи экологии имеют первостепенное значение. Важным этапом решения такого рода задач является разработка и анализ математических моделей различных экологических систем. Одной из основных задач экологии на современном этапе является изучение структуры и функционирования природных систем, поиск общих закономерностей. Большое влияние на экологию оказала математика, способствующая становлению математической экологии, особенно такие её разделы, как теория дифференциальных уравнений, теория устойчивости и теория оптимального управления.

    В данной работе предложены некоторые выкладки по анализу модели Hassell \comment{А как фамилия по-русски пишется? Michael Patrick Hassell?}. Его модель широко используется в качестве общей модели динамики популяции с дискретным временем и наличием конкуренции за ресурсы и внутривидовой конкуренции. 

    \comment{https://en.wikipedia.org/wiki/Michael\_Hassell}

    \comment{https://royalsocietypublishing.org/doi/10.1098/rsos.182178\#d3e228}

    \comment{https://www.hindawi.com/journals/ddns/2020/8148634/}

    \comment{https://www.jstor.org/stable/3863}

    \comment{а что тут еще писать? Refactor this text!}
