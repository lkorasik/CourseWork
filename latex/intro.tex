\section{Введение}

    Задачи экологии имеют первостепенное значние. Важно научиться применять методы для анализа математических моделей различных экологических систем. 
    
    Одна из основных задач экологии --- изучение структуры системы и то, как она функционирует, поиск закономерностей. В качестве инструмента для анализа систем можно использовать методы из различных разделов математики, в частности нелинейеной динамики.
    
    В данной работе предложены некоторые выкладки по анализу дискретной модели Хасселя. Подобные модели широко используется в качестве общих моделей динамики популяции с дискретным временем и наличием конкуренции за ресурсы и внутривидовой конкуренции. Так же их используют для исследования различных явлений в динамике популяций.

    \comment{Нврн у нас есть эффект Алли}

    \comment{https://royalsocietypublishing.org/doi/10.1098/rsos.182178\#d3e228}

    \comment{https://www.hindawi.com/journals/ddns/2020/8148634/}

    \comment{https://www.jstor.org/stable/3863}

    \comment{https://www.imperial.ac.uk/people/m.hassell/publications.html}

    