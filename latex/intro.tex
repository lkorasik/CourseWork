\section{Введение}

    В настоящее время задачи экологии имеют большое значение. Важно научиться применять методы для анализа математических моделей различных экологических систем. 
    
    Одна из основных задач экологии --- изучение структуры системы и то, как она функционирует, поиск закономерностей. В качестве инструмента для анализа систем можно использовать методы из различных разделов математики, в частности нелинейеной динамики.
    
    В данной работе предложены некоторые выкладки по анализу дискретной модели Хасселя с эффектом Олли. Подобные модели широко используется в качестве общих моделей динамики популяции с дискретным временем и наличием конкуренции за ресурсы и внутривидовой конкуренции. Так же их используют для исследования различных явлений в динамике популяций.

    При написании данной работы использовались следующие источники:
    
    \begin{itemize}
        \item \href{https://royalsocietypublishing.org/doi/10.1098/rsos.182178}{Density-Dependence in Single-Species Populations} \cite{densityDependenceInSingleSpeciesPopulations} 
        
        \item Элементы нелинейной динамики: от порядка к хаосу 
        
        \cite{elementsOfNonlinearDynamic}
        
        \item \href{https://www.tandfonline.com/action/showCitFormats?doi=10.1080%2F10236198.2016.1248426}{Nonsmooth one-dimensional maps: some basic concepts and definitions} \cite{nonsmoothOneDimensionalMapsSomeBasicConceptsAndDefinitions}
        
        \comment{если нет ссылок на эту статью, то удаляй ее отсюда}
        
        \item \href{https://royalsocietypublishing.org/doi/10.1098/rsos.182178}{Inequality in resource allocation and population dynamics models} 
        
        \cite{inequalityInResourceAllocationAndPopulationDynamicsModels} 
        
        \item \href{https://www.hindawi.com/journals/ddns/2020/8148634/}{Chaotic Dynamics and Chaos Control of Hassell-Type Recruitment Population Model} \cite{chaoticDynamicsAndChaosControlOfHassellTypeRecruitmentPopulationModel} 
    \end{itemize}

    