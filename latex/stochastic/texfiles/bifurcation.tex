\subsection{Бифуркационные диаграммы}

    Теперь перейдем к рассмотрению графиков бифуркации моделей с разными видами шумов. На рисунках \ref{bifurcation_x_0_2_a_1_compare_beta_noise}, \ref{bifurcation_x_0_2_a_1_compare_alpha_noise} и \ref{bifurcation_x_0_2_a_1_compare_additional_noise} представлены графики бифуркационных диаграмм для моделей с \(\beta\)-шумом, \(\alpha\)-шумом и аддитивным шумом при параметре \(\beta = 1\). Эти графики похожи на детерминированный случай \ref{bifurcation_x_0_2_a_1_compare_no_noise}. На каждом графике, также как и в исходной модели, существует учсток равновесия, участки с циклами. От вида и интенсивности шума зависит то, насколько сильно отличается график с шумом относительно графика для исходной модели \ref{origin}. 

    Самое сильное влияние на значения оказывает \(\beta\)-шум. А, например, \(\alpha\)-шум оказывает незначительное влияние и график очень похож на детерминированный случай. Данные выводы также подтверждаются результатами, которые мы получили при построении временных рядов.

    \comment{Не понятно про что тут писать. Вроде про все сказали в том семетре.}

    \begin{figure}
        \centering
        \includegraphics[width=\textwidth]{stochastic/images/bifurcation_x_0_2_a_1_compare_no_noise.jpg}
        
        \captionsetup{justification=centering}
        \caption{Бифуркационная диаграмма для модели (\ref{origin})}
        \label{bifurcation_x_0_2_a_1_compare_no_noise}
    \end{figure}

    \begin{figure}
        \centering
        \includegraphics[width=\textwidth]{stochastic/images/bifurcation_x_0_2_a_1_compare_alpha_noise.jpg}
        
        \captionsetup{justification=centering}
        \caption{Бифуркационная диаграмма для модели (\ref{alpha_chaos}), \(\varepsilon = 0.01\)}
        \label{bifurcation_x_0_2_a_1_compare_alpha_noise}
    \end{figure}

    \begin{figure}
        \centering
        \includegraphics[width=\textwidth]{stochastic/images/bifurcation_x_0_2_a_1_compare_beta_noise.jpg}
        
        \captionsetup{justification=centering}
        \caption{Бифуркационная диаграмма для модели (\ref{beta_chaos}), \(\varepsilon = 0.01\)}
        \label{bifurcation_x_0_2_a_1_compare_beta_noise}
    \end{figure}

    \begin{figure}
        \centering
        \includegraphics[width=\textwidth]{stochastic/images/bifurcation_x_0_2_a_1_compare_additional_noise.jpg}
        
        \captionsetup{justification=centering}
        \caption{Бифуркационная диаграмма для модели (\ref{additive_chaos}), \(\varepsilon = 0.01\)}
        \label{bifurcation_x_0_2_a_1_compare_additional_noise}
    \end{figure}
