\subsection{Стохастические диаграммы}

    Далее представлено случайное возмущение на бифуркационной диаграмме с разными видами шумов. На рисунках \ref{bifurcation_x_0_2_a_1_compare_alpha_noise}, \ref{bifurcation_x_0_2_a_1_compare_beta_noise} и \ref{bifurcation_x_0_2_a_1_compare_additional_noise} представлены графики бифуркационных диаграмм для моделей (\ref{alpha_chaos}), (\ref{beta_chaos}) и (\ref{additive_chaos}). Во всех моделях зафиксировано значение параметра \(\alpha = 1\). На каждом графике, так же как и в исходной модели, существует участок равновесия, участки с циклами и участок с хаотическим поведением системы. От вида и интенсивности шума зависит то, насколько \comment{А вот тут я не понял правку}.

    Видно, что \(\alpha\)-шум 

    Можно заметить, что самое сильное влияние на значения оказывает \(\beta\)-шум. А, например, \(\alpha\)-шум оказывает незначительное влияние и диаграмма очень похожа на диаграмму детерминированного случая. Данные выводы также подтверждаются результатами, которые мы получили при построении временных рядов.

    \begin{figure}
        \centering
        \subfloat[для модели (\ref{origin})]{
            \includegraphics[width=0.55\textwidth]{stochastic/images/bifurcation_x_0_2_a_1_compare_no_noise.jpg}
            \label{bifurcation_x_0_2_a_1_compare_no_noise}
        }  
        \subfloat[для модели (\ref{alpha_chaos})]{
            \includegraphics[width=0.55\textwidth]{stochastic/images/bifurcation_x_0_2_a_1_compare_alpha_noise.jpg}
            \label{bifurcation_x_0_2_a_1_compare_alpha_noise}
        }
        
        \subfloat[для модели (\ref{beta_chaos})]{
            \includegraphics[width=0.55\textwidth]{stochastic/images/bifurcation_x_0_2_a_1_compare_beta_noise.jpg}
            \label{bifurcation_x_0_2_a_1_compare_beta_noise}
        }
        \subfloat[для модели (\ref{additive_chaos})]{
            \includegraphics[width=0.55\textwidth]{stochastic/images/bifurcation_x_0_2_a_1_compare_additional_noise.jpg}
            \label{bifurcation_x_0_2_a_1_compare_additional_noise}
        }
            
        \caption{Бифуркационная диаграмма при \(\varepsilon = 0.01\)}
    \end{figure}

    % \begin{figure}
    %     \centering
    %     \includegraphics[width=\textwidth]{stochastic/images/bifurcation_x_0_2_a_1_compare_no_noise.jpg}
        
    %     \captionsetup{justification=centering}
    %     \caption{Бифуркационная диаграмма для модели (\ref{origin})}
    %     \label{bifurcation_x_0_2_a_1_compare_no_noise}
    % \end{figure}
