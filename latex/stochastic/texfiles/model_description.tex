\subsection{Описание вносимого возмущения}

        После завершения детерминированного анализа можно перейти к стохастическому анализу. В модели (\ref{origin}) появляется слагаемое, которое отвечает за случайные возмущения. Оно имеет следующий вид: \(\varepsilon \xi\), где \(\varepsilon\) --- интенсивность шума, а \(\xi\) --- нормальная случайная величина. Это слагаемое может быть внесено аддитивно в правую часть уравнения или же в параметр модели. 
        
        Каждый вид шума несет свой биологический смысл. Например, увеличение численности популяции может происходить неравномерно из-за внешних обстоятельств, такому поведению системы соответствует \(\alpha\)-шум. \(\beta\)-шум показывает, что среда обитания также может в зависимости от времени предоставлять разное количество ресурсов для поддержания численности популяции. И наконец, аддитивный шум показывает, что рассматриваемая среда обитания необязательно должна быть изолирована от окружающего мира, может наблюдаться миграция особей.
        
        Отображения для разных видов шума примет формы, которые перечислены ниже.

        \begin{enumerate}
            \item \(\alpha\)-шум

                \begin{equation}
                    \label{alpha_chaos}
                    x_{t + 1} = \frac{(\alpha + \varepsilon \xi) x_t^2}{(\beta + x_t)^6}
                \end{equation}
    
            \item \(\beta\)-шум
    
                \begin{equation}
                    \label{beta_chaos}
                    x_{t + 1} = \frac{\alpha x_t^2}{(\beta + \varepsilon \xi + x_t)^6}
                \end{equation}
    
            \item Аддитивный шум
    
                \begin{equation}
                    \label{additive_chaos}
                    x_{t + 1} = \frac{\alpha x_t^2}{(\beta + x_t)^6} + \varepsilon \xi
                \end{equation}
        \end{enumerate}