\section{Заключение}

    В работе представлены результаты компьютерного моделирования и анализа динамики численности популяции модели Хасселя. В рамках анализа с помощью методов численного моделирования были построены временные ряды, бифуркционные диаграммы, лестница Ламерея, показатель Ляпунова, карта режимов, графики критической интенсивности и графики функции стохастической чувствительности. Были найдены и изучены зоны сосуществования различных аттракторов. 

    Для визуализации и вычислений использовались Python, mathplotlib, SymPy, NumPy, GeoGebra, Wolfram Mathematica и MatLab.
