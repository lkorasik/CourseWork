\section{Заключение}

    В работе представлены результаты компьютерного моделирования и анализа динамики численности популяции модели Хасселя. 
    
    В рамках детерминированного анализа с помощью методов численного моделирования были построены временные ряды, бифуркционные диаграммы, лестница Ламерея, показатель Ляпунова, карта режимов, бассейны притяжения. Были найдены и изучены зоны сосуществования различных аттракторов. 

    При изучении влияния случайного возмущения были расммотрены три вида шума. Построены стохастичесике диаграммы, зависимотсии матоематического ожидания и дисперсии, зависимоть функции стохастической чувствительности аттракторов, полосы рассеивания. Найдены зависимости критической интенсивности необходимые для возникновения переходов с невырожденных аттракторов на вырожденное равновесие. Построена зависимость евклидовой метрики, а также метрики Махаланобис.

    Для визуализации и вычислений использовались Python, mathplotlib, SymPy, NumPy, GeoGebra, Wolfram Mathematica и MatLab.
