\section{Введение}

    Наверное, что-то про модель Хасселя надо сказать.

\section{Что-то после введения}

    Математическая запись модели:

    \[x_{t+1} = \frac{\alpha x_t^2}{(\beta + x_t)^6}\]

    где \(\alpha, \beta\) - параметры. Зафиксируем параметр \(\alpha = 1\). Параметр \(\beta\) изменяется в диапазоне \([0; 0.6]\)    

    \[x = \frac{\alpha x^2}{(\beta + x)^6}\]
    
    \[1 = \frac{\alpha x}{(\beta + x)^6}\]

    \[\alpha x = (\beta + x)^6\]

    Аналитически это уравнение не решается. Для нахождения корней будем использовать метод Ньютона.

    В зависимости от конкретных значений параметра \(\beta\) данное уравнение может иметь от одного до трех корней.
    
    *графики*

\section{Итоги}