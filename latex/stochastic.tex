\section{Стохастический анализ}

    \subsection{Описание изменений}

        После завершения детерминированного анализа можно перейти к стохастическому анализу. В модели (\ref{origin}) появляется слагаемое, которое отвечает за шум. Это слагаемое имеет следующий вид: \(\varepsilon \xi\), где \(\varepsilon\) --- интенсивность шума, а \(\xi\) --- нормальная случайная величина.

        Каждый вид шума несет свой биологический смысл. Например, увеличение численности популяции может происходить неравномерно из-за внешних обстоятельств, такому поведению системы соответствует \(\alpha\)-шум. \(\beta\)-шум показывает, что среда обитания также может в зависимости от времени предоставлять разное количество ресурсов для поддержкания численности популяции. И накоенц, аддитивный шум показывает, что рассматриваемая среда обитания может быть не изолирована от окружающего мира, можно наблюдать миграцию особоей.

        Как я сказал выше, в модели (\ref{origin}) можно рассматривать три вида возможного шума: \(\alpha\)-шум, \(\beta\)-шум и аддитивный шум. Каждый вид шума несет свой биологический смысл.

        \comment{А что на счет комбинации шумов? Нечто может влиять на enviroment capacity и на growth rate одновременно.}

        Вот так будут выглядеть формулы с разными видами шума.

        \(\alpha\)-шум: 

        \begin{equation}
            \label{alpha_chaos}
            x_{t + 1} = \frac{(\alpha + \varepsilon \xi) x_t^2}{(\beta + x_t)^6}
        \end{equation}

        \(\beta\)-шум:

        \begin{equation}
            \label{beta_chaos}
            x_{t + 1} = \frac{\alpha x_t^2}{(\beta + \varepsilon \xi + x_t)^6}
        \end{equation}

        Аддитивный шум:

        \begin{equation}
            \label{additive_chaos}
            x_{t + 1} = \frac{\alpha x_t^2}{(\beta + x_t)^6} + \varepsilon \xi
        \end{equation}

    \subsection{Временные ряды}

        С помощью временных рядов продемонстрируем как различные виды шума влияют на поведение системы. Рассмотрим сразу все возможные варианты шума. 

        На рисунке \ref{time_series_x_0_06_a_1_b_0_56} изображено поведение модели без добавления каких-либо шумов. Мы видим, что значения с теченеием времени стабилизируются. Численность популяции фактически осатется неизменной на протяжении всего оставшегося времени.

        \begin{figure}
            \centering
            \includegraphics[width=\textwidth]{stochastic/time_series_x_0_06_a_1_b_0_56.jpg}

            \captionsetup{justification=centering}
            \caption{Временной ряд модели (\ref{origin}) при \(\beta = 0.56, \alpha = 1, x_0 = 0.06\)}
            \label{time_series_x_0_06_a_1_b_0_56}
        \end{figure}

        Теперь давайте добавим \(\alpha\)-шум, \(\beta\)-шум, аддитивный шум в нашу модель и рассмотрим соответсвенно рисунки \ref{time_series_x_0_06_a_1_b_0_56_alpha_chaos_epsilon_0_004}, \ref{time_series_x_0_06_a_1_b_0_56_beta_chaos_epsilon_0_004} и \ref{time_series_x_0_06_a_1_b_0_56_additive_chaos_epsilon_0_004}. Все варианты рассматриваются с одной и той же интенсивностью шума \(\varepsilon = 0.004\). 
        
        Мы видим, что при одинаковой интенсивности шума его вид влияет на величину разброса значений численности популяции. \comment{Приплети сюда дисперсию} И если раньше численность популяции стабилизировалась и переставала хоть сколько-нибудь меняться, то сейчас численность постоянно всгеда колеблется, но как можно заметить, она меняется в рамках некоторого коридора значений. Численность популяции в общем то не растет и не уменьшается на какую-то значительную величину. Но такое поведение наблюдается не всегда.

        \comment{Мысль: малый шум будет оказывать незначительное влияние, а большой - огромное.}

        \comment{Можно взять оригианльный временной ряд и на него свеху наложить временной ряд с шумом.}

        \comment{Хорошая фраза: переход вызванный индуцированным шумом}

        \begin{figure}
            \centering
            \includegraphics[width=\textwidth]{stochastic/time_series_x_0_06_a_1_b_0_56_alpha_chaos_epsilon_0_004.jpg}
        
            \captionsetup{justification=centering}
            \caption{Временной ряд модели (\ref{origin}) при \(\beta = 0.56, \alpha = 1, x_0 = 0.06, \varepsilon = 0.004\)}
            \label{time_series_x_0_06_a_1_b_0_56_alpha_chaos_epsilon_0_004}
        \end{figure}

        \begin{figure}
            \centering
            \includegraphics[width=\textwidth]{stochastic/time_series_x_0_06_a_1_b_0_56_beta_chaos_epsilon_0_004.jpg}
        
            \captionsetup{justification=centering}
            \caption{Временной ряд модели (\ref{origin}) при \(\beta = 0.56, \alpha = 1, x_0 = 0.06, \varepsilon = 0.004\)}
            \label{time_series_x_0_06_a_1_b_0_56_beta_chaos_epsilon_0_004}
        \end{figure}

        \begin{figure}
            \centering
            \includegraphics[width=\textwidth]{stochastic/time_series_x_0_06_a_1_b_0_56_additive_chaos_epsilon_0_004.jpg}
        
            \captionsetup{justification=centering}
            \caption{Временной ряд модели (\ref{origin}) при \(\beta = 0.56, \alpha = 1, x_0 = 0.06, \varepsilon = 0.004\)}
            \label{time_series_x_0_06_a_1_b_0_56_additive_chaos_epsilon_0_004}
        \end{figure}

        Для того чтоб продемонстрировать другое возможное поведение, давайте увеличим интенсивность шума. Пускай \(\varepsilon = 0.04\). Рассмотрим рисунок \ref{time_series_x_0_06_a_1_b_0_56_beta_chaos_epsilon_0_04_fall}. Здесь мы видим ситуацию, когда шум оказал негативное влияние на численность популяции. Но опять же все не так просто. Если мы еще раз запустим алгоритм, который просчитывает нашу модель, то мы увидим, что популяции удалось выжить на протяжении анализируемого интервала времени. Данный пример проиллюстрирован на картинке \ref{time_series_x_0_06_a_1_b_0_56_beta_chaos_epsilon_0_04_alive}. Аналогичные эффекты можно наблюдать и при других видах шума.

        \comment{Примерно в \(t = 50\) обе поплцяии были близки к вымиранию, но одна выжила, а вторая нет.}

        \begin{figure}
            \centering
            \includegraphics[width=\textwidth]{stochastic/time_series_x_0_06_a_1_b_0_56_beta_chaos_epsilon_0_04_fall.jpg}
        
            \captionsetup{justification=centering}
            \caption{Временной ряд модели (\ref{origin}) при \(\beta = 0.56, \alpha = 1, x_0 = 0.06, \varepsilon = 0.04\)}
            \label{time_series_x_0_06_a_1_b_0_56_beta_chaos_epsilon_0_04_fall}
        \end{figure}

        \begin{figure}
            \centering
            \includegraphics[width=\textwidth]{stochastic/time_series_x_0_06_a_1_b_0_56_beta_chaos_epsilon_0_04_alive.jpg}
        
            \captionsetup{justification=centering}
            \caption{Временной ряд модели (\ref{origin}) при \(\beta = 0.56, \alpha = 1, x_0 = 0.06, \varepsilon = 0.04\)}
            \label{time_series_x_0_06_a_1_b_0_56_beta_chaos_epsilon_0_04_alive}
        \end{figure}

        Анализируя вышесказанное можно сказать, что при добавлении в модель случайных событий ее поведение становится непредсказуемым. Одно незначительное изменение может кардинально повлиять на ход развития событий. 

        "Одно рисовое зернышко может склонить чашу весов" (с) какой-то мультик.

        \comment{рассматривать как в том семестре все возможные варианты поведения системы в зависимости от начальной точки думаю не надо}

    \subsection{Бифуркация}

        Теперь перейдем к рассмотрению графиков бифуркации с разными видами шумов. Все тоже самое, только картинка не такая четкая.

        \begin{figure}
            \centering
            \includegraphics[width=\textwidth]{stochastic/bifurcation_x_0_2_a_1_compare_no_noise.jpg}
        
            \captionsetup{justification=centering}
            \caption{График бифуркации без шума}
            \label{bifurcation_x_0_2_a_1_compare_no_noise}
        \end{figure}

        \begin{figure}
            \centering
            \includegraphics[width=\textwidth]{stochastic/bifurcation_x_0_2_a_1_compare_beta_noise.jpg}
        
            \captionsetup{justification=centering}
            \caption{График бифуркации с \(\beta\)-шумом, \(\varepsilon = 0.01\)}
            \label{bifurcation_x_0_2_a_1_compare_beta_noise}
        \end{figure}

        \begin{figure}
            \centering
            \includegraphics[width=\textwidth]{stochastic/bifurcation_x_0_2_a_1_compare_alpha_noise.jpg}
        
            \captionsetup{justification=centering}
            \caption{График бифуркации с \(\alpha\)-шумом, \(\varepsilon = 0.01\)}
            \label{bifurcation_x_0_2_a_1_compare_alpha_noise}
        \end{figure}

        \begin{figure}
            \centering
            \includegraphics[width=\textwidth]{stochastic/bifurcation_x_0_2_a_1_compare_additional_noise.jpg}
        
            \captionsetup{justification=centering}
            \caption{График бифуркации с аддитивным шумом, \(\varepsilon - 0.01\)}
            \label{bifurcation_x_0_2_a_1_compare_additional_noise}
        \end{figure}

        \comment{напиши что-нибдь про бифуркацю}

    \subsection{Матожидание}

        \comment{напиши что-нибдь про матожидание и циклическое матожидание}

    \subsection{Дисперсия}

        \comment{напиши что-нибдь про дисперсию и циклическую дисперсию}

    \subsection{Функция стохастический чувствительности}
    
        Чтож, а на десерт у нас функция стохастический чувствительности.

        Функция стохастический чувствительности это инструмент, который показывает... Если интересны математические подробности --- иди в статью Crises, noise, and tipping in the Hassell population model.

        Давайте посмотрим на график \ref{bifurcation_x_0_2_a_1_beta_chaos_fss}. Красными линиями нарисована функция стохастический чувствительности. \comment{сформулируй зачем она тут вообще нужна}. 

        \begin{figure}
            \centering
            \includegraphics[width=\textwidth]{stochastic/bifurcation_x_0_2_a_1_beta_noise_fss.jpg}
        
            \captionsetup{justification=centering}
            \caption{График бифуркации с \(\beta\)-шумом.}
            \label{bifurcation_x_0_2_a_1_beta_chaos_fss}
        \end{figure}

        Рассмотрим участок от \(\beta \approx 0.45\) до \(\beta \approx 0.48\), он изображен на рисунке \ref{bifurcation_x_0_2_a_1_beta_chaos_fss_segment_stable}. Мы видим, что зачения графика бифуркации почти всегда находятся в коридоре, границами которого являются значеня ФСЧ. Этот коридор строится по правилу трех сигм.Такой подход гарантирует, что почти все значения будут находится в этом интервале, что собственно мы и наблюдаем.

        \begin{figure}
            \centering
            \includegraphics[width=\textwidth]{stochastic/bifurcation_x_0_2_a_1_beta_noise_fss_segment_stable.jpg}
        
            \captionsetup{justification=centering}
            \caption{}
            \label{bifurcation_x_0_2_a_1_beta_chaos_fss_segment_stable}
        \end{figure}

        На участках с k-циклами и хаосом (рисунки \ref{bifurcation_x_0_2_a_1_beta_chaos_fss_segment_2_cycle}, \ref{bifurcation_x_0_2_a_1_beta_chaos_fss_segment_chaos_down} и \ref{bifurcation_x_0_2_a_1_beta_chaos_fss_segment_chaos_up]}) будет наблюдаться аналогичная ситуация: значения лежат в коридоре, ограниченом значениями ФСЧ.

        \comment{и для цикла шаг поменьше, чтобы плавнее было по параметру
        и для других шумов так же. если строишь уже для отчета картинки, то на увеличении для зон - равновесий и циклом шаг по параметру для стохастической диаграммы тоже чаще, чтобы не было отдельных отрезков
        а в остальном прекрасно}

        \begin{figure}
            \centering
            \includegraphics[width=\textwidth]{stochastic/bifurcation_x_0_2_a_1_beta_noise_fss_segment_2_cycle.jpg}
        
            \captionsetup{justification=centering}
            \caption{}
            \label{bifurcation_x_0_2_a_1_beta_chaos_fss_segment_2_cycle}
        \end{figure}

        \begin{figure}
            \centering
            \includegraphics[width=\textwidth]{stochastic/bifurcation_x_0_2_a_1_beta_noise_fss_segment_chaos_up.jpg}
        
            \captionsetup{justification=centering}
            \caption{}
            \label{bifurcation_x_0_2_a_1_beta_chaos_fss_segment_chaos_up}
        \end{figure}

        \begin{figure}
            \centering
            \includegraphics[width=\textwidth]{stochastic/bifurcation_x_0_2_a_1_beta_noise_fss_segment_chaos_down.jpg}
        
            \captionsetup{justification=centering}
            \caption{}
            \label{bifurcation_x_0_2_a_1_beta_chaos_fss_segment_chaos_down}
        \end{figure}

        \comment{напиши что-нибдь про ФСЧ}

    \subsection{График ФСЧ}

        \comment{напиши что-нибдь про график фсч}
        